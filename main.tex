\documentclass{beamer}
\usepackage{amsmath}
\usepackage{tikz-cd}
\usepackage{textcomp}
\usepackage{lmodern}
\usepackage[T1]{fontenc}

\usetheme{Berlin}

\newcommand{\cat}[1]{\mathsf{#1}}

\title{An Introduction to Abelian Categories}
\author{Gabriel Antonio Videtta}
\date{May 2025}

\begin{document}

\begin{frame}
    \titlepage
\end{frame}

\begin{frame}{Preliminary steps (i)}
    The intuition for abelian categories comes from the
    behaviour of a kind of
    mathematical object which is found
    everywhere in each science: \textbf{vector spaces}. \bigskip


    Vector spaces are considered to be well understood
    and have interesting categorical properties.
\end{frame}

\begin{frame}{Preliminary steps (ii)}
    First of all, we will denote with $\cat{Vect}_K$ the category
    of vector spaces over the field $K$, whose objects are vector spaces and
    morphisms are linear maps. \bigskip


    We will denote with $\cat{FinDimVect}_K$ the subcategory
    of $\cat{Vect}_K$ which contains only finite dimensional
    $K$-vector spaces.
\end{frame}

\begin{frame}{Preliminary steps (iii)}
    Vector spaces have a peculiar property: initial objects are
    isomorphic to final objects! This tells use that $0$ (the
    zero dimensional vector space) is a special object.
    
    \begin{definition}[Zero object]
        Let $\mathcal{C}$ be a category. We say $0$ is a
        \textbf{zero object} if it's both an initial and
        a final object.
    \end{definition}
\end{frame}

\begin{frame}{Preliminary steps (iv)}
    Moreover, the hom-set $\hom(V, W)$ has an additional
    structure: it's not just a set, it has a natural structure
    of a vector space as well! \medskip

    We can sum linear maps ($f + g$), multiply a linear map by
    a scalar ($\lambda f$), and all these operations behave
    ``bilinearly'' with the composition ($\circ$):
    \[
        (f + g) \circ h = f \circ h + g \circ h,
    \]
    \[
        f \circ (g + h) = f \circ g + f \circ h,
    \]
    \[
        (\lambda f) \circ g = \lambda (f \circ g) = f \circ (\lambda g).
    \]

    This gives rise to an important definition...
\end{frame}

\begin{frame}{Preliminary steps (v)}
    \begin{definition}[Enriched categories]
        Let $\mathcal{C}$ be a category. We say that $\mathcal{C}$
        is a category \textbf{enriched over a category $\mathcal{D}$}
        if the hom-sets of $\mathcal{C}$ are objects from
        $\mathcal{D}$. \medskip


        We restrict $\mathcal{D}$ to be a monoidal category (e.g.,
        cartesian category, $\cat{Ab}$, $\cat{Vect}_K$, $\cat{Mod}_R$).
    \end{definition}

    Therefore, we can say that $\cat{Vect}_K$ is enriched over itself!
\end{frame}

\begin{frame}{Preliminary steps (vi)}
    \begin{definition}[Preadditive category]
        Let $\mathcal{C}$ be a category. We say that $\mathcal{C}$
        is a \textbf{preadditive category} if it's enriched
        over the category of abelian groups ($\cat{Ab}$).
    \end{definition}
\end{frame}

\end{document}