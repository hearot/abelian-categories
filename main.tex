\documentclass{beamer}
\usepackage{amsmath}
\usepackage{textcomp}
\usepackage{lmodern}
\usepackage[T1]{fontenc}
\usepackage{quiver}

\usetheme{Berlin}

\newtheorem{proposition}{Proposition}

\DeclareMathOperator{\id}{id}

\newcommand{\ZZ}{\mathbb{Z}}
\newcommand{\cat}[1]{\mathsf{#1}}
\newcommand{\ob}{\cat{ob}}

\title{An Introduction to Abelian Categories}
\author{Gabriel Antonio Videtta}
\date{May 2025}

\begin{document}

\begin{frame}
    \titlepage
\end{frame}

\begin{frame}{Preliminary steps (i)}
    The intuition for abelian categories comes from the
    behaviour of a kind of
    mathematical object which is found
    everywhere in each science: \textbf{vector spaces}. \bigskip


    Vector spaces are considered to be well understood
    and have interesting categorical properties.
\end{frame}

\begin{frame}{Preliminary steps (ii)}
    First of all, we will denote with $\cat{Vect}_K$ the category
    of vector spaces over the field $K$, whose objects are vector spaces and
    morphisms are linear maps. \bigskip


    We will denote with $\cat{FinDimVect}_K$ the subcategory
    of $\cat{Vect}_K$ which contains only finite dimensional
    $K$-vector spaces.
\end{frame}

\begin{frame}{Preliminary steps (iii): zero objects}
    Vector spaces have a peculiar property: initial objects are
    isomorphic to final objects! This tells use that $0$ (the
    zero dimensional vector space) is a special object.
    
    \begin{definition}[Zero object]
        Let $\mathcal{C}$ be a category. We say $0$ is a
        \textbf{zero object} if it's both an initial and
        a final object.
    \end{definition}
\end{frame}

\begin{frame}{Zero morphisms}
    Zero objects also allow for the
    definition of zero morphisms, which will be regarded as the
    ``trivial morphism'' from $A$ to $B$ (e.g., sending all elements
    of $A$ into a ``null element'', if the category is an
    algebraic one). \medskip

    \begin{definition}
        Let $A$ and $B$ be two objects and let $0$ be a zero
        object of $\mathcal{C}$. Then we define $0_{AB}$ as
        follows
        \[
            0_{AB} \triangleq \, ?_{B} \, \circ \, !_{A}, \qquad \text{where} \quad ?_B : 0 \to B, \quad\; !_A : A \to 0.
        \]
    \end{definition}

    The definition is independent of the zero object $0$ and is
    in this sense ``unique''.
\end{frame}

\begin{frame}{Preliminary steps (iv)}
    Moreover, the hom-set $\hom(V, W)$ has an additional
    structure: it's not just a set, it has a natural structure
    of a vector space as well! \medskip

    We can sum linear maps ($f + g$), multiply a linear map by
    a scalar ($\lambda f$), and all these operations behave
    ``bilinearly'' with the composition ($\circ$):
    \[
        (f + g) \circ h = f \circ h + g \circ h,
    \]
    \[
        f \circ (g + h) = f \circ g + f \circ h,
    \]
    \[
        (\lambda f) \circ g = \lambda (f \circ g) = f \circ (\lambda g).
    \]

    This gives rise to an important definition...
\end{frame}

\begin{frame}{Preliminary steps (v): enriched categories}
    \begin{definition}[Enriched categories]
        Let $\mathcal{C}$ be a category. We say that $\mathcal{C}$
        is a category \textbf{enriched over a monoidal category $(\mathcal{D}, \otimes)$}
        if the hom-sets of $\mathcal{C}$ are objects from
        $\mathcal{D}$ and if the composition of morphisms makes
        the composition $\circ$ bilinear over $\otimes$, namely:

        \[
            (F \otimes G) \circ H = (F \circ H) \otimes (G \circ H),
        \]
        \[
            F \circ (G \otimes H) = (F \circ G) \otimes (F \circ H).
        \]
    \end{definition}

    Therefore, we can say that $\cat{Vect}_K$ is enriched over itself!
\end{frame}

\begin{frame}{Preliminary steps (vi): preadditive categories}
    Recall that an abelian group is a monoid which allows inverses
    and satisfies the law of commutativity. For example, a vector
    space $V$ is itself an abelian group. \medskip

    \begin{definition}[Preadditive category]
        Let $\mathcal{C}$ be a category. We say that $\mathcal{C}$
        is a \textbf{preadditive category} if it's enriched
        over the category of abelian groups ($\cat{Ab}$).
    \end{definition} \smallskip

    In short, a preadditive category is such that its morphisms can
    be added and subtracted in a way that respects composition.
\end{frame}

\begin{frame}{Preliminary steps (vii): modules and relationship with $\cat{Ab}$}
    Before we properly discuss abelian categories, let's introduce
    the last fundamental algebraic structure we're going to talk about
    in this seminary: \textbf{modules}. \medskip
    
    Modules are pretty much ``vector spaces over a ring'': they have the
    same axioms as a vector space, except they are built over a ring,
    which does not have to allow inverses. \medskip

    Notice that abelian groups are $\ZZ$-modules, where:
    \[
        n \cdot x := \underbrace{x + x + \ldots + x}_{n \text{ times}}.
    \]

    This fact will result useful later on.
\end{frame}

\begin{frame}
    Products and coproducts behave in the same way in a
    pre-additive category, as shown below. \smallskip

    \begin{proposition}
        Let $\mathcal{C}$ be a preadditive category. Then
        products and coproducts are isomorphic to one
        another in $\mathcal{C}$.
    \end{proposition}

    We only prove that products are also coproducts; the other part of
    the statement is proved similarly.
\end{frame}

\begin{frame}[fragile]
    \begin{proof}
        \only<1>{
            Let $A$ and $B$ be two objects in $\mathcal{C}$ and
            let $(C := A \times B, \pi_A : C \to A, \pi_B : C \to B)$ be a product of
            $A$ and $B$. We shall determine two morphisms
            $\iota_A : A \to C$ and $\iota_B : B \to C$ such that
            $(C, \iota_A, \iota_B)$ is also a coproduct of $A$ and
            $B$. \medskip

            In doing so, we strive to get some ``injections'' of $A$ and $B$
            into $C$. A way of doing that is to use the universal
            property of $C$ and extend the following morphisms
            to two morphisms $\iota_A$, $\iota_B : A, \, B \to C$:

            \begin{enumerate}
                \item $\id_A$, $0_{AB} \leadsto \iota_A$,
                \item $0_{BA}$, $\id_B \leadsto \iota_B$.
            \end{enumerate}
        }

        \only<2>{
        $\iota_A$ and $\iota_B$ yield the following commutative diagram:

        \[\begin{tikzcd}[sep=large, ampersand replacement=\&]
            \& A \\
            A \& {A \times B} \& B \\
            \& B
            \arrow["{\operatorname{id}_A}", from=1-2, to=2-1]
            \arrow["{\iota_A}", from=1-2, to=2-2]
            \arrow["{0_{AB}}", from=1-2, to=2-3]
            \arrow["{\pi_A}", from=2-2, to=2-1]
            \arrow["{\pi_B}"', from=2-2, to=2-3]
            \arrow["{0_{BA}}", from=3-2, to=2-1]
            \arrow["{\iota_B}", from=3-2, to=2-2]
            \arrow["{\operatorname{id}_B}", from=3-2, to=2-3]
        \end{tikzcd}\]
        }

        \only<3>{
        Let's now prove that $(C, \iota_A, \iota_B)$ is a coproduct. Let
        $D$ be an object from $\mathcal{C}$ and let $f$, $g : A$, $B \to D$
        be morphisms. \medskip

        Let's define $h_{f, g} : C \to D$ such that:
        \[
            h_{f, g} = f \circ \pi_A + g \circ \pi_B. 
        \]

        $h_{f, g}$ will play the role of the ``connecting morphism'' from
        $C$ to $D$.
        }

        \only<4>{
        We then expect that $h_{\iota_A, \iota_B}$ -- the
        connecting morphism generated by the ``injections'' -- will behave
        as the identity on $C$. \medskip


        Since the following identities hold:
        \begin{eqnarray*}
            \pi_A \circ h_{\iota_A, \iota_B} = \pi_A \circ (\iota_A \circ \pi_A + \iota_B \circ \pi_B) = \pi_A, \\
            \pi_B \circ h_{\iota_A, \iota_B} = \pi_B \circ (\iota_A \circ \pi_A + \iota_B \circ \pi_B) = \pi_B.
        \end{eqnarray*}
        then $h_{\iota_A, \iota_B}$ is indeed $\id_C$ by the universal property of the product.
        }

        \only<5>{
        Let's now prove that $h_{f, g}$ is the unique morphism which makes the following diagram commute:
        \[\begin{tikzcd}[sep=large, ampersand replacement=\&]
            A \&\& B \\
            \& C \\
            \& D
            \arrow["{\iota_A}"', from=1-1, to=2-2]
            \arrow["f", curve={height=12pt}, from=1-1, to=3-2]
            \arrow["{\iota_B}", from=1-3, to=2-2]
            \arrow["g"', curve={height=-12pt}, from=1-3, to=3-2]
            \arrow["{h_{f, g}}", from=2-2, to=3-2]
        \end{tikzcd}\]
        The commutativity can easily be proved by hand, since $0_{AB}$ and $0_{BA}$ are the zeroes of
        $\hom(A, B)$ and $\hom(B, A)$, respectively.
        }

        \only<6>{
        On the other hand, uniqueness is proved as follows:
        \begin{eqnarray*}
            h_{f, g} - h'
            &=& (h_{f, g} - h') \circ h_{\iota_A, \iota_B} \\
            &=& (h_{f, g} - h') \circ (\iota_A \circ \pi_A + \iota_B \circ \pi_B) \\
            &=& \ldots \\
            &=& 0_{C, D}.
        \end{eqnarray*}
        }

        \alt<6>{\qedhere}{\phantom\qedhere}
    \end{proof}
\end{frame}

\begin{frame}
    We're missing just two ingredients for the definition of pre-abelian categories:
    \textbf{kernels} and \textbf{cokernels}. Let's derive kernels from an example,
    and cokernels will be defined as the dual of kernels. \medskip
    
    
    In linear algebra, given a linear map $f : V \to W$, we define $\ker f$ as follows:
    \[
        \ker f \triangleq \{ v \in V \mid f(v) = w \}.
    \]
    Of course we're not allowed to defined kernels as sets, but only as morphisms.
    An elementary property of $\ker f$ is that $\ker f$ is a subspace of $V$, hence
    there exists a natural injection map $\iota : \ker f \to V$ such that:

    \[\begin{tikzcd}[ampersand replacement=\&]
        {\ker f} \& V \& W
        \arrow["\iota", hook, from=1-1, to=1-2]
        \arrow["f", from=1-2, to=1-3]
    \end{tikzcd}, \qquad f \circ \iota = 0_{VW}. \]
\end{frame}

\begin{frame}
    It is then natural to define the kernel of $f : A \to B$ as the ``biggest morphism $k$''
    that annihilates $f$. \medskip

    \begin{definition}
        Let $f : A \to B$ be a morphism. Then a kernel $k$ of $f$ is a morphism
        $k : K \to A$ such that:
        \begin{enumerate}
            \item $f \circ k = 0_{KB}$,
            \item If $k' : K' \to A$ is a morphism such that $f \circ k' = 0_{K'B}$, then
                there exists a unique morphism $\iota_{K'} : K' \to K$ such that
                $k' = k \circ \iota_{K'}$.
        \end{enumerate}
    \end{definition}
\end{frame}

\begin{frame}
    Dually, a cokernel of $f : A \to B$ is the ``smallest morphism $j$'' that
    $f$ annihilates. \medskip

    \begin{definition}
        Let $f : A \to B$ be a morphism. Then a cokernel $j$ of $f$ is a morphism
        $j : B \to J$ such that:
        \begin{enumerate}
            \item $j \circ f = 0_{AJ}$,
            \item If $j' : B \to J'$ is a morphism such that $j' \circ f = 0_{AJ'}$, then
                there exists a unique morphism $\pi_{J'} : J \to J'$ such that
                $j' = \pi_{J'} \circ j$.
        \end{enumerate}
    \end{definition}
\end{frame}

\begin{frame}
    \begin{definition}
        A pre-additive category is pre-abelian if it allows kernels and cokernels for
        every morphism $f$ and permits products for a finite family of objects.
    \end{definition}
\end{frame}

\end{document}