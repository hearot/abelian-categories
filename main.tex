\documentclass{beamer}
\usepackage{amsmath}
\usepackage{textcomp}
\usepackage{lmodern}
\usepackage[T1]{fontenc}
\usepackage{quiver}

\usetheme{Berlin}

\newtheorem{proposition}{Proposition}

\DeclareMathOperator{\id}{id}

\newcommand{\ZZ}{\mathbb{Z}}
\newcommand{\cat}[1]{\mathsf{#1}}
\newcommand{\ob}{\cat{ob}}

\title{An Introduction to Abelian Categories}
\author{Gabriel Antonio Videtta}
\date{May 2025}

\begin{document}

\begin{frame}
    \titlepage
\end{frame}

\begin{frame}{Preliminary steps (i)}
    The intuition for abelian categories comes from the
    behaviour of a kind of
    mathematical object which is found
    everywhere in each science: \textbf{vector spaces}. \bigskip


    Vector spaces are considered to be well understood
    and have interesting categorical properties.
\end{frame}

\begin{frame}{Preliminary steps (ii)}
    First of all, we will denote with $\cat{Vect}_K$ the category
    of vector spaces over the field $K$, whose objects are vector spaces and
    morphisms are linear maps. \bigskip


    We will denote with $\cat{FinDimVect}_K$ the subcategory
    of $\cat{Vect}_K$ which contains only finite dimensional
    $K$-vector spaces.
\end{frame}

\begin{frame}{Preliminary steps (iii): zero objects}
    Vector spaces have a peculiar property: initial objects are
    isomorphic to final objects! This tells use that $0$ (the
    zero dimensional vector space) is a special object.
    
    \begin{definition}[Zero object]
        Let $\mathcal{C}$ be a category. We say $0$ is a
        \textbf{zero object} if it's both an initial and
        a final object.
    \end{definition}
\end{frame}

\begin{frame}{Preliminary steps (iv)}
    Moreover, the hom-set $\hom(V, W)$ has an additional
    structure: it's not just a set, it has a natural structure
    of a vector space as well! \medskip

    We can sum linear maps ($f + g$), multiply a linear map by
    a scalar ($\lambda f$), and all these operations behave
    ``bilinearly'' with the composition ($\circ$):
    \[
        (f + g) \circ h = f \circ h + g \circ h,
    \]
    \[
        f \circ (g + h) = f \circ g + f \circ h,
    \]
    \[
        (\lambda f) \circ g = \lambda (f \circ g) = f \circ (\lambda g).
    \]

    This gives rise to an important definition...
\end{frame}

\begin{frame}{Preliminary steps (v): enriched categories}
    \begin{definition}[Enriched categories]
        Let $\mathcal{C}$ be a category. We say that $\mathcal{C}$
        is a category \textbf{enriched over a monoidal category $(\mathcal{D}, \otimes)$}
        if the hom-sets of $\mathcal{C}$ are objects from
        $\mathcal{D}$ and if the composition of morphisms makes
        the composition $\circ$ bilinear over $\otimes$, namely:

        \[
            (F \otimes G) \circ H = (F \circ H) \otimes (G \circ H),
        \]
        \[
            F \circ (G \otimes H) = (F \circ G) \otimes (F \circ H).
        \]
    \end{definition}

    Therefore, we can say that $\cat{Vect}_K$ is enriched over itself!
\end{frame}

\begin{frame}{Preliminary steps (vi): preadditive categories}
    Recall that an abelian group is a monoid which allows inverses
    and satisfies the law of commutativity. For example, a vector
    space $V$ is itself an abelian group. \medskip

    \begin{definition}[Preadditive category]
        Let $\mathcal{C}$ be a category. We say that $\mathcal{C}$
        is a \textbf{preadditive category} if it's enriched
        over the category of abelian groups ($\cat{Ab}$).
    \end{definition} \smallskip

    In short, a preadditive category is such that its morphisms can
    be added and subtracted in a way that respects composition.
\end{frame}

\begin{frame}{Preliminary steps (vii): modules and relationship with $\cat{Ab}$}
    Before we properly discuss abelian categories, let's introduce
    the last fundamental algebraic structure we're going to talk about
    in this seminary: \textbf{modules}. \medskip
    
    Modules are pretty much ``vector spaces over a ring'': they have the
    same axioms as a vector space, except they are built over a ring,
    which does not have to allow inverses. \medskip

    Notice that abelian groups are $\ZZ$-modules, where:
    \[
        n \cdot x := \underbrace{x + x + \ldots + x}_{n \text{ times}}.
    \]

    This fact will result useful later on.
\end{frame}

\begin{frame}
    Products and coproducts behave in the same way in a
    pre-additive category, as shown below.

    \begin{proposition}
        Let $\mathcal{C}$ be a preadditive category. Then
        products and coproducts are isomorphic to one
        another in $\mathcal{C}$.
    \end{proposition}
\end{frame}

\begin{frame}
    \begin{proof}
        Let $A$ and $B$ be two objects in $\mathcal{C}$ and
        let $(C := A \times B, \pi_A : C \to A, \pi_B : C \to B)$ be a product of
        $A$ and $B$. We shall determine two morphisms
        $\iota_A : A \to C$ and $\iota_B : B \to C$ such that
        $(C, \iota_A, \iota_B)$ is also a coproduct of $A$ and
        $B$. \medskip

        In doing so, we strive to get some ``injections'' of $A$ and $B$
        into $A \times B$. A way of doing that is to use the universal
        property of $A \times B$ and extend the following morphisms
        to two morphisms $\iota_A$, $\iota_B : A, \, B \to C$:

        \begin{enumerate}
            \item $\id_A$, $0_{AB} \leadsto \iota_A$,
            \item $0_{BA}$, $\id_B \leadsto \iota_B$.
        \end{enumerate}
    \end{proof}
\end{frame}

\begin{frame}[fragile]
    \begin{proof}
        $\iota_A$ and $\iota_B$ yield the following commutative diagram:

        \[\begin{tikzcd}[sep=2.25em]
            & A \\
            A & {A \times B} & B \\
            & B
            \arrow["{\operatorname{id}_A}", from=1-2, to=2-1]
            \arrow["{\iota_A}", from=1-2, to=2-2]
            \arrow["{0_{AB}}", from=1-2, to=2-3]
            \arrow["{\pi_A}", from=2-2, to=2-1]
            \arrow["{\pi_B}"', from=2-2, to=2-3]
            \arrow["{0_{BA}}", from=3-2, to=2-1]
            \arrow["{\iota_B}", from=3-2, to=2-2]
            \arrow["{\operatorname{id}_B}", from=3-2, to=2-3]
        \end{tikzcd}\]
    \end{proof}
\end{frame}

\begin{frame}
    \begin{proof}
        Let's now prove that $(C, \iota_A, \iota_B)$ is a coproduct. Let
        $D$ be an object from $\mathcal{C}$ and let $f$, $g : A$, $B \to D$
        be morphisms. \medskip

        Let's define $h : C \to D$ such that:
        \[
            h = f \circ \pi_A + g \circ \pi_B. 
        \] 
    \end{proof}
\end{frame}

\end{document}