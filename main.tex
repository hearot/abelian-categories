\documentclass{beamer}
\usepackage{amsmath}
\usepackage{tikz-cd}
\usepackage{textcomp}
\usepackage{lmodern}
\usepackage[T1]{fontenc}

\usetheme{Berlin}

\newcommand{\cat}[1]{\mathsf{#1}}

\title{An Introduction to Abelian Categories}
\author{Gabriel Antonio Videtta}
\date{May 2025}

\begin{document}

\begin{frame}
    \titlepage
\end{frame}

\begin{frame}{Preliminary steps (i)}
    The intuition for abelian categories comes from the
    behaviour of a kind of
    mathematical object which is found
    everywhere in each science: \textbf{vector spaces}. \bigskip


    Vector spaces are considered to be well understood
    and have interesting categorical properties.
\end{frame}

\begin{frame}{Preliminary steps (ii)}
    First of all, we will denote with $\cat{Vect}_K$ the category
    of vector spaces over the field $K$, whose objects are vector spaces and
    morphisms are linear maps. \bigskip


    We will denote with $\cat{FinDimVect}_K$ the subcategory
    of $\cat{Vect}_K$ which contains only finite dimensional
    $K$-vector spaces.
\end{frame}

\begin{frame}{Preliminary steps (iii)}
    Vector spaces have a peculiar property: initial objects are
    isomorphic to final objects! \medskip
    
    Moreover $\cat{FinDimVect}_K$ is self-dual (even though
    it's not a ``natural duality''!), finite dimensional
    vector spaces share even more good properties.
\end{frame}

\end{document}